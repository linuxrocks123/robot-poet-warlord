\documentclass[aspectratio=169]{beamer}
\usetheme{CambridgeUS}
\begin{document}

\title{Robot-Poet-Warlord: An Introduction}
\subtitle{CSCI1301: Computer Science I}
\author{Patrick Simmons}
\institute{Georgia College}
\date{\today}

\begin{frame}
  \titlepage
\end{frame}

\begin{frame}{The Game}
  \begin{itemize}
  \item Your code manages a team of robots.
  \item The goal of this team is to defeat all other teams of robots
    by destroying all enemy robots.
  \item The environment your robots operate in is a 2D grid,
    represented in the API as a 2D array.
  \item You can't see the entire 2D grid unless you use your radio to
    get a copy of it each turn.
  \end{itemize}
\end{frame}

\begin{frame}{The Rules}
  \begin{itemize}
  \item Your robots get 20 skill points apiece when created by the
    simulator, but you shouldn't assume this since you can create
    robots yourself; the createRobot method you implement tells you
    how many skill points you get with a parameter.
  \item Your code chooses how to distribute these skill points when
    your robot is created.
  \item The simulator will give your robots unique ID numbers.  You
    can use these to create different robot types at the creation of
    the world (with the \% operator).
  \item You could also use the Random class, of course.
  \end{itemize}
\end{frame}

\begin{frame}{Skills}
  \begin{itemize}
  \item Attack, Defense, Power, Charge
  \item Attack: base chance of hitting, limit to how much power you
    can put into a single attack
  \item Defense: base chance of resisting attack, limit to how much
    power you can put into defense
  \item Power: maximum power you can use for any number of actions
    in a turn
  \item Charge: how much power you ``regenerate'' each turn.  Also,
    10 times your charge skill is your maximum charge, and also your
    maximum health.
  \end{itemize}
\end{frame}

\begin{frame}{Combat (from attacker's perspective)}
  \begin{itemize}
  \item Melee: Your Attack Skill + Power you put into attack is your
    ``modified attack skill''.
  \item Ranged: (1/2 your attack skill) + Power you put into attack is
    your ``modified attack skill''.
  \item Capsule: separate rules
  \item Defense: Their Defense skill + power they put into defense
    last turn is their ``modified defense skill''.
  \item Chance of attack hitting is 50\% + (Your modified attack
    skill - their modified defense skill) * 10
  \item Damage of attack if hit: \textbf{power you put into attack}\\
    So, it's important to have a highish power skill just for
    attacking, but remember you can't put any more power into an
    attack than your attack skill.
  \end{itemize}
\end{frame}

\begin{frame}{Looking Around}
  \begin{itemize}
  \item getVisibleNeighborhood(): gives you the portion of the world
    you can see.  You can see a number of squares equal to your
    defense skill in front of you, in back of you, and to your right
    and left.  This cross is then circumscribed with a square, so
    your visible area is a square 2*(defense skill)+1 squares
    centered on you.
  \item If you're near the edge of the world, the visible
    neighborhood will not include anything past the edge of the
    world, so getVisibleNeighborhood() will return a smaller
    rectangle.
  \item getWorld(): spend 3 power to get a copy of the entire world
  \end{itemize}
\end{frame}

\begin{frame}{Moving Around: Self Explanatory}
  \begin{itemize}
  \item move(int steps, Direction way)
  \item pick\_up\_capsule(GridCell adjacent\_cell)
  \item drop\_capsule(GridCell adjacent\_cell, int power\_of\_capsule)
  \end{itemize}
\end{frame}

\begin{frame}{Moving Around: This Is Important}
  \begin{itemize}
  \item I wrote three methods for you to use for pathfinding.
  \item public static GridCell findNearestAlly()
  \item public static GridCell[] findShortestPath()
  \item private static GridCell[] findShortestPathInternal()
  \item The arguments to these methods are described in the
    documentation.
  \end{itemize}
\end{frame}

\begin{frame}{Pathfinding}
  \begin{itemize}
  \item findShortestPathInternal() implements Dijkstra's algorithm for
    pathfinding.
  \item This algorithm and similar ones are covered later in the CS
    major sequence; for now, you can just know that this gives you the
    shortest path from A to B, where you choose A and B.
  \item The exact API for the non-private methods is described in the
    documentation.
  \item I \emph{strongly encourage} you to use these methods instead
    of trying to implement your own code to figure out whether it is
    possible for you to get to a particular square, and, if so, what
    the shortest path is.
  \end{itemize}
\end{frame}

\begin{frame}{Construction}
  \begin{itemize}
  \item Can construct capsules, walls, forts, and robots.
  \item If you begin constructing a wall, fort, or robot, you need to
    specify an empty cell to construct in.  That cell then becomes
    ``blocked'', and no one can move onto it unless the robot doing
    the construction is killed.
  \end{itemize}
\end{frame}

\begin{frame}{Radio}
  \begin{itemize}
  \item Each message costs 1 power to send to nearest neighbor, or 2
    to send to all allies.
  \item The messages are each an array of 64 bytes.  What you do with
    those 64 bytes is entirely up to you.
  \item If you're going to use the radio, you should probably define a
    ``protocol'', where each robot identifies who it is some way
    (perhaps by position) in the message.  No ID of the sender is
    included, though you are guaranteed only to receive messages from
    allies.
  \item You never receive messages that you yourself send.
  \item Radio also allows you to scan a visible enemy and discover its
    statistics within your ``visible range'' and to get a copy of the
    world map.
  \item The scanning also works on allies if you want to use it for
    some reason.
  \end{itemize}
\end{frame}
    

\end{document}
